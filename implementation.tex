\label{sec:implement}

\subsection{Tracing memory access}
\subsection{Tracking cache invalidations}
\subsection{Reporting false sharing}
\subsection{Predicting false sharing}
\subsection{Predicting performance improvement}

Fixing false sharing problems can be non-trivial, even with precise information about a particular false sharing problem. Several problems may occur. 
The first problem is that some false sharing problems can be insignificant. 
For example, Sheriff reports few false sharing problems. But fixing them brings negligible performance benefit, such as word\_count or reverse\_index applications in Phoenix example. 

The second problem is that fixing false sharing problem may even slow down the program because of that excessive memory consumption. For example, padding falsely-shared objects will introduce some . 

What is the idea of predicting performance improvement after fixes?  

We predict how much performance slowdown that can be caused . 

\cheetah{} provides a upper bound on performance improvement after fixes. 

It is note that sampling based approaches can actually 
 