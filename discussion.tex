\section{Discussion}

\label{sec:discuss}
\subsection{Effectiveness}

Limitation. 
We have to setup the hardware platform at first. 
Then Applications needs to run long enough in order to get meaningful results. 
However, this is not a problem since we may not care about a program's performance if it only runs a few seconds. 

Different sampling rate. 

\Cheetah{} may miss those applications that won't have large performance impact. 

\subsection{Performance Overhead}

\Cheetah{}, with the support of lightweight IBS, can identify cache line false sharing efficiently and effectively. However, its runtime overhead can still be as high as 10-20\%, mainly due to the mechanism of AMD IBS. AMD IBS samples every kind of instructions, including arithmetic instructions (e.g., add, sub, mul, and div) and logic instructions (e.g., comp and test), which are useless for analyzing false sharing. Instead, \cheetah{} only requires sampling memory accesses. Because of this hardware limitation, \cheetah{} needs to filter out useless samples with a software method. Such software method needs \cheetah{} to receive samples and check corresponding bits to test if these samples access memory or not; these processes incur high overhead. Hopefully, the hardware can overcome this limitations in the future by only sampling memory loads and stores.

\subsection{Effectiveness}
