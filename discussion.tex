\section{Discussion}

\label{sec:discuss}

This section addresses some possible concerns related to \Cheetah{}. 

\paragraph{Hardware Dependence.} \cheetah{} is an approach that relies on the hardware PMU unit to sample memory accesses. Thus, users should setup the hardware correctly  Currently, we have implemented the support for . It takes time to setup the memory sampling. In the future, we are planning to provide more support.
 
\paragraph{Performance Overhead.} On average, \Cheetah{} only introduces 7\% performance overhead for all evaluated applications. However, \cheetah{} does introduce more than 20\% overhead for two applications with a large number of threads because \cheetah{} should setup hardware registers for every thread. Although creating a large number of threads in an application is not normal, we expect that we can only setup once  with better hardware support. We also expect that the overhead can be further reduced by only sampling memory accesses, but not other non-related instructions such as arithmetic instructions and logic instructions. Hopefully, the hardware can overcome these two limitations in the future.

\paragraph{Effectiveness.} 

We expect that 


we see that \Cheetah{} We still expect that the overhead can be further reduced in the future.  

Limitation. 
We have to setup the hardware platform at first. 
Then Applications needs to run long enough in order to get meaningful results. 
However, this is not a problem since we may not care about a program's performance if it only runs a few seconds. 

Different sampling rate. 

Hardware related performance overhead 


\Cheetah{} may miss those applications that won't have large performance impact. 

\subsection{Effectiveness}
