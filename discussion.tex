\section{Discussion}

\label{sec:discuss}

This section addresses some possible concerns related to \Cheetah{}. 

\paragraph{Hardware Dependence.} \cheetah{} is an approach that relies on hardware PMUs to sample memory accesses. To use \cheetah{}, users should setup the driver to enable the PMU-based sampling beforehand. Afterward, they can connect to the \cheetah{} library by calling only two APIs: one API is to setup PMU-based registers, while the other handles every sampled memory access, with less than 5 lines of code change. 
%Users should add PMU initialization at startup of a program or in the beginning of every thread, with adding less than 5 lines of code in total. 
%Currently, we have implemented the hardware support for the AMD Opteron machine and will add the support for Intel PEBS-related machines. 

\paragraph{Performance Overhead.} On average, \Cheetah{} only introduces 7\% performance overhead for all evaluated applications. However, \cheetah{} does introduce more than 20\% overhead for two applications that having a large number of threads, because \cheetah{} should monitor thread creations and setup hardware registers for every thread. However, this should not be of large concern when using \cheetah{}. First, the creation of a large number of threads in an application is atypical. Secondly, we expect this overhead could be further reduced with improved hardware support. 
%We also expect that the overhead can be further reduced by only sampling memory accesses, but not non-related instructions such as arithmetic instructions and logic instructions.  \cheetah{} may further reduce the performance overhead with better hardware support in the future.

\paragraph{Effectiveness.} \Cheetah{} effectively detects false sharing problems that occur in the current execution, and have high impact on performance. For effective detection, \Cheetah{} requires programs to run sufficiently long, perhaps more than few seconds. This should not be a problem for long-running applications, which are the primary targets of optimizations. 
