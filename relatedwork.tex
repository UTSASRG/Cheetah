\label{sec:relatedwork}

\subsection{False Sharing Detection}
Jayasena et. al. proposes to identify possible memory access patterns that can be caused by false sharing by utilizing a machine learning approach~\cite{mldetect}. Their approach collects events about memory accesses, data caches, TLBs, interactions among cores, and resources stalls, and derives potential memory patterns that can cause false sharing. However, their approach only identify a subset of false sharing problems that are found by \Predator{}~\cite{Predator} and DARWIN~\cite{openmp}. DARWIN utilizes Data Event Ad-
dress Register (DEAR) on the Itanium 2 processor to sample memory reads~\cite{Wicaksono11detectingfalse, openmp}. Different with \Cheetah{}, DARWIN detects cache coherence events in the first execution and then identifies data structures that can possibly cause false sharing in the second execution.  However, DARWIN cannot present precise information about falsely-shared cache lines. Intel's Performance Tuning Utility (PTU) is a commercialized tool to utilize Intel's (Precise Event Based Sampling registers (PEBS) to detect false sharing. But it cannot pinpoint the root cause of false sharing objects since it does not intercept those memory allocations. 
All of these tools suffer the same shortcoming: they cannot differentiate false sharing , leaving much burden to programmers for fixing false sharing problems. 