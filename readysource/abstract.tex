%False sharing occurs when multiple threads, running on different cores with separate caches, access logically independent words in the same cache line. 

%Multicore processors are ubiquitous across the full computing spectrum, from phones, desktops, to high-end servers. 
False sharing is a notorious performance problem that may occur in multithreaded programs when they are running on ubiquitous multicore hardware. It can dramatically degrade the performance by up to an order of magnitude, significantly hurting the scalability. Identifying false sharing in complex programs is challenging. Existing tools either incur significant performance overhead or do not provide adequate information to guide code optimization.
%On the one hand, it is necessary to monitor memory accesses in order to report problematic instructions and data objects, 
%accurately and distinguish them with true sharing, but that usually incurs very high performance overhead. On the other hand, existing lightweight techniques have limited precisions in detecting false sharing and often provide inadequate information to guide optimization.
%that can dramatically degrade the performance of multithreading programs by up to an order of magnitude. 
%The hardware trend, including adding more cores into the same machine, introducing the Non-Uniform-Memory-Access (NUMA) architecture, or increasing the size of a cache line, will further degrade the performance of false sharing problems, making the task of detecting more urgent. 
%Existing detection tools have different types of shortcomings: most tools cannot report precise information about false sharing; some may introduce too much performance overhead; some have a lot of limitations on applications or the environment; all tools cannot assess how much performance improvement by fixing a specific false sharing problem. 

\sloppy
To address these problems, we develop \cheetah{}, a profiler that detects false sharing both efficiently and effectively. \cheetah{} leverages the lightweight hardware performance monitoring units (PMUs) that are available in most modern CPU architectures to sample memory accesses. \cheetah{} develops the first approach to quantify the optimization potential of false sharing instances without actual fixes, based on the latency information collected by PMUs. \cheetah{} precisely reports false sharing and provides insightful optimization guidance for programmers, while adding less than $7\%$ runtime overhead on average. \Cheetah{} is ready for real deployment. 

%Unlike prior work, \cheetah{} quantitatively assesses the potential gains of fixing each false sharing instance, so programmers can focus on the most severe problems only. %\cheetah{} provides the most comprehensive information that can assist users to fix false sharing problems. 
%We evaluate \cheetah{} on two benchmark suites (Phoenix and PARSEC), and multiple real applications. 
%According to our evaluation, \Cheetah{} can detect false sharing, . 
%Guided by \Cheetah{}, we obtain significant performance improvement. %ready to be used in deployment environment because of its negligible performance overhead (3\%) and its effectiveness.  
%\category{C.4}{Performance of systems}{Measurement techniques, Performance attributes}
%\category{D.2.8}{Metrics}{Performance measures}

%\terms
%Mesurement, Performance

%\keywords{Multithreading, False Sharing, Performance Prediction, Address Sampling, Lightweight Profiling}
