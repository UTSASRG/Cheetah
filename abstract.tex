%False sharing occurs when multiple threads, running on different cores with separate caches, access logically independent words in the same cache line. 
False sharing can dramatically degrade the performance of multithreading programs by up to an order of magnitude. 
%The hardware trend, including adding more cores into the same machine, introducing the Non-Uniform-Memory-Access (NUMA) architecture, or increasing the size of a cache line, will further degrade the performance of false sharing problems, making the task of detecting more urgent. 
Existing detection tools face with the following shortcomings: some may introduce too much performance overhead; some can only work on a specific type of applications; all tools cannot predict how much performance improvement by fixing a specific false sharing problem. 

This paper presents a detection tool--\cheetah{}--to detect false sharing tool both {\bf efficiently} and {\bf effectively}. \cheetah{} leverages hardware performance counters that are available on most existing modern hardware to sample memory accesses, thus it introduces negligible performance overhead and can be utilized in deployment environment. Different with previous work, \cheetah{} can predict the upper bound of performance improvement after fixes so that programmers can focus on important problems only. \cheetah{} provides the most comprehensive information that can assist users to fix false sharing problems. \cheetah{} has been evaluated over a range of applications, including two benchmark suites (Phoenix and PARSEC) and multiple high performance applications. \Cheetah{} is ready to eliminate all false sharing problems.  